\documentclass[10,a5paper]{article}
\usepackage[T2A]{fontenc}
\usepackage[utf8]{inputenc}
%koi8-r]{inputenc}
\usepackage[english,russian]{babel}
\usepackage{amsmath}
\usepackage{amsfonts}
\usepackage{amssymb}
\usepackage[top=1cm,bottom=0.5cm,left=1cm, right=1cm]{geometry}

% \usepackage{color}
\usepackage[usenames,dvipsnames,svgnames,table]{xcolor}
\usepackage[colorlinks=true, linkcolor=Maroon]{hyperref}
\pagestyle{empty}


\newcommand*{\mc}{}
\def\mc#1#{\mathcoloraux{#1}}
\newcommand*{\mathcoloraux}[3]{%
  \protect\leavevmode
  \begingroup
    \color#1{#2}#3%
  \endgroup
}

\usepackage{wrapfig}


\usepackage{graphicx}
\usepackage{physics}

\usepackage{textcomp}

\usepackage{comment}
\newcommand{\fref}[1]{Рис.~\ref{#1}}
\newcommand{\tref}[1]{Таб.~\ref{#1}}
\newcommand{\eref}[1]{(\ref{#1})}


\renewcommand{\labelenumii}{\arabic{enumii}.}
\renewcommand{\thesubsubsection}{\arabic{subsubsection}}

\usepackage{listings}

\setcounter{secnumdepth}{6}
\setcounter{tocdepth}{4}

\title{Массивы}

\begin{document}
\section{Массивы}
\subsubsection{Рекурсия}
Повторим предыдущую тему: нарисуйте двоичное дерево.
\subsubsection{Границы массива}
что произойдёт, если выйти за границы массива?
\subsubsection{Случайные числа}
Заполните массив из 24 целых чисел случайным образом от нуля до ста.
Используйте функцию \texttt{random(100)}.
\subsubsection{Максимум}
Найдите максимум в массиве из случайных чисел.
\subsubsection{Минимум}
Найдите минимум в массиве из случайных чисел.
\subsubsection{Поиск элемента}
В массиве из случайных чисел найдите позицию, в которой находится число
\texttt{50}. Если такой позиции нет, напишите \texttt{-1}. Выведите на
экран, сколько раз приходится сравнивать элементы массива.
\subsubsection{Перестановка}
Напишите процедуру, которая будет переставлять два элемента.
\subsubsection{Сортировка}
Отсортируйте массив в порядке возрастания. Сколько перестановок
потребовалось?
\subsubsection{Бинарный поиск}
Сортируйте массив в порядке возрастания.
При сортировке массива сохраните начальные позиции чисел в массиве.
\subsubsection{Стек}
Стек --- это структура данных в виде <<стопки>> однотипных переменных.
Пользователь имеет доступ только до верхнего элемента.
Напишите процедуру \texttt{push(x: integer)}, которая кладёт новый
элемент \texttt{x} в <<стопку>>, и функцию \texttt{pop: integer},
которая возвращает верхний элемент <<стопки>> и возвращает его значение.
\end{document}
